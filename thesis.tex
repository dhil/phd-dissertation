%% 12pt font size, PhD thesis, LFCS, print twosided, new chapters on right page
\documentclass[12pt,phd,lfcs,twoside,openright,logo,leftchapter,normalheadings]{infthesis}
\shieldtype{0}

%% Packages
\usepackage[utf8]{inputenc}   % Enable UTF-8 typing
\usepackage[british]{babel}   % British English
\usepackage[breaklinks]{hyperref} % Interactive PDF
\usepackage{url}
\usepackage[sort&compress,square,numbers]{natbib}    % Bibliography
\usepackage{breakurl}
\usepackage{amsmath}          % Mathematics library
\usepackage{amssymb}          % Provides math fonts
\usepackage{amsthm}           % Provides \newtheorem, \theoremstyle, etc.
\usepackage{mathtools}
\usepackage{pkgs/mathpartir}  % Inference rules
\usepackage{stmaryrd}         % semantic brackets
\usepackage{array}
\usepackage{float}            % Float control
\usepackage{caption,subcaption}  % Sub figures support
\usepackage[T1]{fontenc}      % Fixes font issues
%\usepackage{lmodern}
\usepackage[activate=true,
    final,
    tracking=true,
    kerning=true,
    spacing=true,
    factor=1100,
    stretch=10,
    shrink=10]{microtype}
\usepackage{enumerate}        % Customise enumerate-environments
\usepackage{xcolor}           % Colours
\usepackage{xspace}           % Smart spacing in commands.
\usepackage{tikz}
\usetikzlibrary{fit,calc,trees,positioning,arrows,chains,shapes.geometric,%
    decorations.pathreplacing,decorations.pathmorphing,shapes,%
    matrix,shapes.symbols,intersections}

\SetProtrusion{encoding={*},family={bch},series={*},size={6,7}}
              {1={ ,750},2={ ,500},3={ ,500},4={ ,500},5={ ,500},
               6={ ,500},7={ ,600},8={ ,500},9={ ,500},0={ ,500}}

\SetExtraKerning[unit=space]
    {encoding={*}, family={bch}, series={*}, size={footnotesize,small,normalsize}}
    {\textendash={400,400}, % en-dash, add more space around it
     "28={ ,150}, % left bracket, add space from right
     "29={150, }, % right bracket, add space from left
     \textquotedblleft={ ,150}, % left quotation mark, space from right
     \textquotedblright={150, }} % right quotation mark, space from left

%%
%% Load macros.
%%
%%
%% Calculi names.
%%
\newcommand{\Links}{Links\xspace}
\newcommand{\CoreLinks}{\ensuremath{\mathsf{CoreLinks}}\xspace}
\newcommand{\BCalc}{\ensuremath{\lambda_{\mathsf{b}}}\xspace}
\newcommand{\BCalcRec}{\ensuremath{\lambda_{\mathsf{b}+\mathsf{rec}}}\xspace}

%%
%% Calculi terms and types type-setting.
%%
\newcommand{\revto}{\ensuremath{\leftarrow}}

\newcommand{\dec}[1]{\mathsf{#1}}
\newcommand{\keyw}[1]{\mathbf{#1}}
\newcommand{\Handle}{\keyw{handle}}
\newcommand{\ShallowHandle}{\ensuremath{\keyw{handle}^\dagger}}
\newcommand{\With}{\keyw{with}}
\newcommand{\Let}{\keyw{let}}
\newcommand{\Rec}{\keyw{rec}}
\newcommand{\In}{\keyw{in}}
\newcommand{\Do}{\keyw{do}}
\newcommand{\Return}{\keyw{return}}
\newcommand{\Val}{\keyw{val}}
\newcommand{\Case}{\keyw{case}}
\newcommand{\If}{\keyw{if}}
\newcommand{\Then}{\keyw{then}}
\newcommand{\Else}{\keyw{else}}
\newcommand{\Absurd}{\keyw{absurd}}
\newcommand{\Record}[1]{\ensuremath{\langle #1 \rangle}}
\newcommand{\Unit}{\Record{}}
\newcommand{\Inl}{\keyw{inl}}
\newcommand{\Inr}{\keyw{inr}}
\newcommand{\Thunk}{\lambda \Unit.}

\newcommand{\Pre}[1]{\mathsf{Pre}(#1)}
\newcommand{\Abs}{\mathsf{Abs}}
\newcommand{\Presence}{\mathsf{Presence}}
\newcommand{\Row}{\mathsf{Row}}
\newcommand{\Type}{\mathsf{Type}}
\newcommand{\Ground}{\mathsf{ground}}

\newcommand{\Comp}{\mathsf{Comp}}
\newcommand{\Effect}{\mathsf{Effect}}
\newcommand{\Handler}{\mathsf{Handler}}

\newcommand{\ZeroType}{0}
\newcommand{\UnitType}{1}
\newcommand{\One}{1}
\newcommand{\Int}{\mathsf{Int}}
\newcommand{\Bool}{\mathsf{Bool}}
\newcommand{\List}{\mathsf{List}}
\newcommand{\Nat}{\mathsf{Nat}}
\newcommand{\Choose}{\dec{Choose}}
\newcommand{\Count}{\dec{count}}
\newcommand{\GenericSearch}{\dec{genericSearch}}
\newcommand{\Predicate}{\dec{Predicate}}
\newcommand{\Point}{\dec{Point}}
\newcommand{\Branch}{\dec{Branch}}
\newcommand{\Get}{\dec{Get}}
\newcommand{\Put}{\dec{Put}}
\newcommand{\Zero}{\dec{Zero}}
\newcommand{\Fail}{\dec{Fail}}

\newcommand{\True}{\mathsf{true}}
\newcommand{\False}{\mathsf{false}}

\newcommand{\eff}{!}
\newcommand{\typ}[2]{#1 \vdash #2}
\newcommand{\typv}[2]{#1 \vdash #2}
\newcommand{\typc}[3]{#1 \vdash #2 \eff #3}

\newcommand{\FTV}{\ensuremath{\mathrm{FTV}}}

\newcommand{\reducesto}[0]{\ensuremath{\leadsto}}
\newcommand{\stepsto}[0]{\ensuremath{\longrightarrow}}
\newcommand{\EC}{\ensuremath{\mathcal{E}}}

%% Handler projections.
\newcommand{\hret}{H^{\mathrm{val}}}
\newcommand{\hval}{\hret}
\newcommand{\hops}{H^{\mathrm{ops}}}
%\newcommand{\hex}{H^{\mathrm{ex}}}
\newcommand{\hell}{H^{\ell}}

\newcommand{\alertbox}[2]{{\par\noindent\small\color{red} \framebox{\parbox{\dimexpr\linewidth-2\fboxsep-2\fboxrule}{\textbf{#1:} #2}}}}
\newcommand{\todo}[1]{\alertbox{TODO}{#1}}
\newcommand{\dhil}[1]{\alertbox{Daniel}{#1}}

%%
%% Labels
%%
\newcommand{\slab}[1]{\textrm{#1}}
\newcommand{\klab}[1]{\textrm{K-#1}}
\newcommand{\semlab}[1]{\textrm{S-#1}}
\newcommand{\tylab}[1]{\textrm{T-#1}}
\newcommand{\mlab}[1]{\text{\scshape{M-#1}}}
\newcommand{\siglab}[1]{\text{\scshape{Sig-#1}}}
\newcommand{\rowlab}[1]{\text{\scshape{R-#1}}}

%%
%% Syntactic categories.
%%
\newcommand{\CatName}[1]{\textrm{#1}}
\newcommand{\CompCat}{\CatName{Comp}}
\newcommand{\ValCat}{\CatName{Val}}
\newcommand{\VarCat}{\CatName{Var}}
\newcommand{\TypCat}{\CatName{Type}}
\newcommand{\TyVarCat}{\CatName{TyVar}}
\newcommand{\KindCat}{\CatName{Kind}}
\newcommand{\RowCat}{\RowCat}

%%
%% Lindley's array stuff.
%%
\newcommand{\ba}{\begin{array}}
\newcommand{\ea}{\end{array}}

\newcommand{\bl}{\ba[t]{@{}l@{}}}
\newcommand{\el}{\ea}


%%
%% Lindley's syntax, reductions, equations, and derivation environments.
%%
\newenvironment{syntax}{\[\ba{@{}l@{\quad}r@{~}c@{~}l@{}}}{\ea\]\ignorespacesafterend}
\newenvironment{reductions}{\[\ba{@{}l@{\qquad}@{}r@{~~}c@{~~}l@{}}}{\ea\]\ignorespacesafterend}

\newenvironment{eqs}{\ba{@{}r@{~}c@{~}l@{}}}{\ea}
\newenvironment{equations}{\[\ba{@{}r@{~}c@{~}l@{}}}{\ea\]\ignorespacesafterend}
\newenvironment{derivation}{\begin{displaymath}\ba{@{}r@{~}l@{}}}{\ea\end{displaymath}\ignorespacesafterend}
\newcommand{\reason}[1]{\quad (\text{#1})}


\newenvironment{smathpar}{\vspace{-3ex}\small\begin{mathpar}}{\end{mathpar}\normalsize\ignorespacesafterend}

%%
%% Defined-as equality
%%
\newcommand{\defas}[0]{\mathrel{\overset{\makebox[0pt]{\mbox{\normalfont\tiny\text{def}}}}{=}}}

%% Information about the title, etc.
% \title{Higher-Order Theories of Handlers for Algebraic Effects}
% \title{Handlers for Algebraic Effects: Applications, Compilation, and Expressiveness}
% \title{Applications, Compilation, and Expressiveness for Effect Handlers}
% \title{Handling Computational Effects}
% \title{Programming Computable Effectful Functions}
% \title{Handling Effectful Computations}
\title{Foundations for Programming and Implementing Effect Handlers}
\author{Daniel Hillerström}

%% If the year of submission is not the current year, uncomment this line and
%% specify it here:
\submityear{2020}

%% Specify the abstract here.
\abstract{%
  An abstract\dots
}

%% Now we start with the actual document.
\begin{document}
\raggedbottom
%% First, the preliminary pages
\begin{preliminary}

%% This creates the title page
\maketitle

%% Acknowledgements
\begin{acknowledgements}
  List of people to thank
  \begin{itemize}
    \item Sam Lindley
    \item John Longley
    \item Christophe Dubach
    \item KC Sivaramakrishnan
    \item Stephen Dolan
    \item Anil Madhavapeddy
    \item Gemma Gordon
    \item Leo White
    \item Andreas Rossberg
    \item Robert Atkey
    \item Jeremy Yallop
    \item Simon Fowler
    \item Craig McLaughlin
    \item Garrett Morris
    \item James McKinna
    \item Brian Campbell
    \item Paul Piho
    \item Amna Shahab
    \item Gordon Plotkin
    \item Ohad Kammar
    \item School of Informatics (funding)
    \item Google (Kevin Millikin, Dmitry Stefantsov)
    \item Microsoft Research (Daan Leijen)
  \end{itemize}
\end{acknowledgements}

%% Next we need to have the declaration.
% \standarddeclaration
\begin{declaration}
  I declare that this thesis was composed by myself, that the work
  contained herein is my own except where explicitly stated otherwise
  in the text, and that this work has not been submitted for any other
  degree or professional qualification except as specified.
\end{declaration}

%% Finally, a dedication (this is optional -- uncomment the following line if
%% you want one).
% \dedication{To my mummy.}
\dedication{\emph{To be or to do}}

% \begin{preface}
% A preface will possibly appear here\dots
% \end{preface}

%% Create the table of contents
\setcounter{secnumdepth}{2} % Numbering on sections and subsections
\setcounter{tocdepth}{1} % Show chapters, sections and subsections in TOC
%\singlespace
\tableofcontents
%\doublespace

%% If you want a list of figures or tables, uncomment the appropriate line(s)
% \listoffigures
% \listoftables
\end{preliminary}

%%%%%%%%%%%%%%%%%%%%%%%%%%%%%%%%%%%
%%          Main content         %%
%%%%%%%%%%%%%%%%%%%%%%%%%%%%%%%%%%%

%%
%% Introduction
%%
\chapter{Introduction}
\label{ch:introduction}
An enthralling introduction\dots
%
Motivation: 1) compiler perspective: unifying control abstraction,
lean runtime, desugaring of async/await, generators/iterators, 2)
giving control to programmers, safer microkernels, everything as a
library.

\section{Thesis outline}
Thesis outline\dots

\section{Typographical conventions}
Explain conventions\dots

\part{Background}
\label{p:background}

\chapter{The state of effectful programming}
\label{ch:related-work}

\section{Type and effect systems}
\section{Monadic programming}

\chapter{Continuations}
\label{ch:continuations}

\section{Zoo of control operators}
Describe how effect handlers fit amongst shift/reset, prompt/control,
callcc, J, catchcont, etc.

\section{Implementation strategies}


\part{Design}

\chapter{A ML-flavoured programming language}
\label{ch:base-language}

In this chapter we introduce a core calculus, \BCalc{}, which we shall
later use as the basis for exploration of design considerations for
effect handlers. This calculus is based on \CoreLinks{} by
\citet{LindleyC12}, which distils the essence of the functional
multi-tier web-programming language
\Links{}~\cite{CooperLWY06}. \Links{} belongs to the
ML-family~\cite{MilnerTHM97} of programming languages as it features
typical characteristics of ML languages such as a static type system
supporting parametric polymorphism and type inference, and its
evaluation semantics is strict. However, \Links{} differentiates
itself from the rest of the ML-family by making crucial use of
\emph{row polymorphism} to support extensible records, variants, and
tracking of computational effects. Thus \Links{} has a rather strong
emphasis on structural types rather than nominal types.

\CoreLinks{} captures all of these properties of \Links{}. Our
calculus \BCalc{} differs in several aspects from \CoreLinks{}. For
example, the underlying formalism of \CoreLinks{} is call-by-value,
whilst the formalism of \BCalc{} is \emph{fine-grain
  call-by-value}~\cite{LevyPT03}, which shares similarities with
A-normal form (ANF)~\cite{FlanaganSDF93} as it syntactically
distinguishes between value and computation terms by mandating every
intermediate computation being named. However unlike ANF, fine-grain
call-by-value remains closed under $\beta$-reduction. The reason for
choosing fine-grain call-by-value as our formalism is entirely due to
convenience. As we shall see in Chapter~\ref{ch:unary-handlers}
fine-grain call-by-value is a convenient formalism for working with
continuations. Another point of difference between \CoreLinks{} and
\BCalc{} is that the former models the integrated database query
sublanguage of \Links{}. We discard the query sublanguage altogether,
and instead focus entirely on the interaction with computational
effects.

\section{Syntax and static semantics}
\label{sec:syntax-base-language}

\section{Type and effect inference}
\dhil{While I would like to detail the type and effect inference, it
  may not be worth the effort. The reason I would like to do this goes
  back to 2016 when Richard Eisenberg asked me about how we do effect
  inference in Links.}

\section{Dynamic semantics}

\chapter{Unary handlers}
\label{ch:unary-handlers}

\section{Deep handlers}
\subsection{Syntax and static semantics}
\subsection{Effect inference}
\subsection{Dynamic semantics}

\section{Parameterised handlers}

\section{Shallow handlers}
\label{ch:shallow-handlers}

\subsection{Syntax and static semantics}
\subsection{Dynamic semantics}

\chapter{N-ary handlers}
\label{ch:multi-handlers}

% \section{Syntax and Static Semantics}
% \section{Dynamic Semantics}
\section{Unifying deep and shallow handlers}

\part{Implementation}

\chapter{Continuation passing styles}
\chapter{Abstract machine semantics}

\part{Expressiveness}
\chapter{Computability, complexity, and expressivness}
\label{ch:expressiveness}
\section{Notions of expressiveness}
Felleisen's macro-expressiveness, Longley's type-respecting
expressiveness, Kammar's typability-preserving expressiveness.

\section{Interdefinability of deep and shallow Handlers}
\section{Encoding parameterised handlers}

\chapter{The asymptotic power of control}
\label{ch:handlers-efficiency}
Describe the methodology\dots
\section{Generic search}
\section{Calculi}
\subsection{Base calculus}
\subsection{Handler calculus}
\section{A practical model of computation}
\subsection{Syntax}
\subsection{Semantics}
\subsection{Realisability}
\section{Points, predicates, and their models}
\section{Efficient generic search with effect handlers}
\subsection{Space complexity}
\section{Best-case complexity of generic search without control}
\subsection{No shortcuts}
\subsection{No sharing}

\chapter{Robustness of the asymptotic power of control}
\section{Mutable state}
\section{Exception handling}
\section{Effect system}

\part{Conclusions}

\chapter{Conclusions}
\label{ch:conclusions}
Some profound conclusions\dots

\chapter{Future Work}
\label{ch:future-work}

%%
%% Appendices
%%
% \appendix

%% If you want the bibliography single-spaced (which is allowed), uncomment
%% the next line.
%\nocite{*}
\singlespace
%\nocite{*}
%\printbibliography[heading=bibintoc]
\bibliographystyle{plainnat}
\bibliography{\jobname}

%% ... that's all, folks!
\end{document}
