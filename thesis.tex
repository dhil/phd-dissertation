%% 12pt font size, PhD thesis, LFCS, print twosided, new chapters on right page
\documentclass[12pt,phd,lfcs,twoside,openright,logo,leftchapter,normalheadings]{infthesis}
\shieldtype{0}

%% Packages
\usepackage[utf8]{inputenc}   % Enable UTF-8 typing
\usepackage[british]{babel}   % British English
\usepackage[breaklinks]{hyperref} % Interactive PDF
\usepackage{url}
\usepackage[sort&compress,square,numbers]{natbib}    % Bibliography
\usepackage{breakurl}
\usepackage{amsmath}          % Mathematics library
\usepackage{amssymb}          % Provides math fonts
\usepackage{amsthm}           % Provides \newtheorem, \theoremstyle, etc.
\usepackage{mathtools}
\usepackage{pkgs/mathpartir}  % Inference rules
\usepackage{stmaryrd}         % semantic brackets
\usepackage{array}
\usepackage{float}            % Float control
\usepackage{caption,subcaption}  % Sub figures support
\DeclareCaptionFormat{underlinedfigure}{#1#2#3\hrulefill}
\captionsetup[figure]{format=underlinedfigure}
\usepackage[T1]{fontenc}      % Fixes font issues
%\usepackage{lmodern}
\usepackage[activate=true,
    final,
    tracking=true,
    kerning=true,
    spacing=true,
    factor=1100,
    stretch=10,
    shrink=10]{microtype}
\usepackage{enumerate}        % Customise enumerate-environments
\usepackage{xcolor}           % Colours
\usepackage{xspace}           % Smart spacing in commands.
\usepackage{tikz}
\usetikzlibrary{fit,calc,trees,positioning,arrows,chains,shapes.geometric,%
    decorations.pathreplacing,decorations.pathmorphing,shapes,%
    matrix,shapes.symbols,intersections}

\SetProtrusion{encoding={*},family={bch},series={*},size={6,7}}
              {1={ ,750},2={ ,500},3={ ,500},4={ ,500},5={ ,500},
               6={ ,500},7={ ,600},8={ ,500},9={ ,500},0={ ,500}}

\SetExtraKerning[unit=space]
    {encoding={*}, family={bch}, series={*}, size={footnotesize,small,normalsize}}
    {\textendash={400,400}, % en-dash, add more space around it
     "28={ ,150}, % left bracket, add space from right
     "29={150, }, % right bracket, add space from left
     \textquotedblleft={ ,150}, % left quotation mark, space from right
     \textquotedblright={150, }} % right quotation mark, space from left

%%
%% Load macros.
%%
%%
%% Defined-as equality
%%
\newcommand{\defas}[0]{\mathrel{\overset{\makebox[0pt]{\mbox{\normalfont\tiny\text{def}}}}{=}}}
\newcommand{\defnas}[0]{\mathrel{:=}}
\newcommand{\simdefas}[0]{\mathrel{\overset{\makebox[0pt]{\mbox{\normalfont\tiny\text{def}}}}{\simeq}}}

%%
%% Some useful maths abbreviations
%%
\newcommand{\C}{\ensuremath{\mathbb{C}}}
\newcommand{\N}{\ensuremath{\mathbb{N}}}
\newcommand{\Z}{\ensuremath{\mathbb{Z}}}
\newcommand{\B}{\ensuremath{\mathbb{B}}}
\newcommand{\BB}[1]{\ensuremath{\mathbf{#1}}}
\newcommand{\CC}{\keyw{ctrl}}
% \newcommand{\Delim}[1]{\ensuremath{\langle\!\!\mkern-1.5mu\langle#1\rangle\!\!\mkern-1.5mu\rangle}}
\newcommand{\Delim}[1]{\ensuremath{\keyw{del}.#1}}
\newcommand{\sembr}[1]{\ensuremath{\llbracket #1 \rrbracket}}

%%
%% Partiality
%%
\newcommand{\pto}[0]{\ensuremath{\rightharpoonup}}

%%
%% Operation arrows
%%
\newcommand{\opto}[0]{\ensuremath{\twoheadrightarrow}}

%%
%% Calculi names.
%%
\newcommand{\Links}{Links\xspace}
\newcommand{\CoreLinks}{\ensuremath{\mathsf{CoreLinks}}\xspace}
\newcommand{\BCalc}{\ensuremath{\lambda_{\mathsf{b}}}\xspace}
\newcommand{\BCalcRec}{\ensuremath{\lambda_{\mathsf{b}+\mathsf{rec}}}\xspace}
\newcommand{\HCalc}{\ensuremath{\lambda_{\mathsf{h}}}\xspace}
\newcommand{\SCalc}{\ensuremath{\lambda_{\mathsf{h^\dagger}}}\xspace}
\newcommand{\HSCalc}{\ensuremath{\lambda_{\mathsf{h^\delta}}}\xspace}
\newcommand{\EffCalc}{\ensuremath{\lambda_{\mathsf{eff}}}\xspace}
\newcommand{\UCalc}{\ensuremath{\lambda_{\mathsf{u}}}\xspace}
\newcommand{\param}{\ensuremath{\ddagger}}

%%
%% Calculi terms and types type-setting.
%%
\newcommand{\revto}{\ensuremath{\leftarrow}}

\newcommand{\dec}[1]{\mathsf{#1}}
\newcommand{\keyw}[1]{\mathbf{#1}}
\newcommand{\Handle}{\keyw{handle}}
\newcommand{\ParamHandle}{\Handle^\param}
\newcommand{\ShallowHandle}{\ensuremath{\keyw{handle}^\dagger}}
\newcommand{\With}{\keyw{with}}
\newcommand{\Let}{\keyw{let}}
\newcommand{\Rec}{\keyw{rec}}
\newcommand{\In}{\keyw{in}}
\newcommand{\Do}{\keyw{do}}
\newcommand{\Return}{\keyw{return}}
\newcommand{\Val}{\keyw{val}}
\newcommand{\Case}{\keyw{case}}
\newcommand{\If}{\keyw{if}}
\newcommand{\Then}{\keyw{then}}
\newcommand{\Else}{\keyw{else}}
\newcommand{\Absurd}{\keyw{absurd}}
\newcommand{\Record}[1]{\ensuremath{\langle #1 \rangle}}
\newcommand{\Op}[1]{\ensuremath{\langle\!\!\langle #1 \rangle\!\!\rangle}}
%\newcommand{\Op}[1]{\ensuremath{\{#1\}}}
\newcommand{\OpCase}[3]{\Op{#1~#2 \opto #3}}
\newcommand{\ExnCase}[2]{\Op{#1~#2}}
\newcommand{\Unit}{\Record{}}
\newcommand{\Inl}{\keyw{inl}}
\newcommand{\Inr}{\keyw{inr}}
\newcommand{\Thunk}{\lambda \Unit.}

\newcommand{\Pre}[1]{\mathsf{Pre}(#1)}
\newcommand{\Abs}{\mathsf{Abs}}
\newcommand{\Presence}{\mathsf{Presence}}
\newcommand{\Row}{\mathsf{Row}}
\newcommand{\Type}{\mathsf{Type}}
\newcommand{\Ground}{\mathsf{ground}}

\newcommand{\Comp}{\mathsf{Comp}}
\newcommand{\Effect}{\mathsf{Effect}}
\newcommand{\Handler}{\mathsf{Handler}}

\newcommand{\ZeroType}{0}
\newcommand{\UnitType}{1}
\newcommand{\One}{1}
\newcommand{\Int}{\mathsf{Int}}
\newcommand{\Float}{\mathsf{Float}}
\newcommand{\Bool}{\mathsf{Bool}}
\newcommand{\List}{\mathsf{List}}
\newcommand{\Nat}{\mathsf{Nat}}
\newcommand{\Choose}{\dec{Choose}}
\newcommand{\Count}{\dec{count}}
\newcommand{\GenericSearch}{\dec{genericSearch}}
\newcommand{\Predicate}{\dec{Predicate}}
\newcommand{\Point}{\dec{Point}}
\newcommand{\Branch}{\dec{Branch}}
\newcommand{\Get}{\dec{Get}}
\newcommand{\Put}{\dec{Put}}
\newcommand{\Zero}{\dec{Zero}}
\newcommand{\Fail}{\dec{Fail}}
\newcommand{\Read}{\dec{Read}}
\newcommand{\Write}{\dec{Write}}
\newcommand{\Char}{\dec{Char}}
\newcommand{\String}{\dec{String}}

\newcommand{\True}{\mathsf{true}}
\newcommand{\False}{\mathsf{false}}

\newcommand{\eff}{!}
\newcommand{\typ}[2]{#1 \vdash #2}
\newcommand{\typv}[2]{#1 \vdash #2}
\newcommand{\typc}[3]{#1 \vdash #2 \eff #3}
\newcommand{\Harrow}{\Rightarrow}

\newcommand{\FTV}{\ensuremath{\mathrm{FTV}}}
\newcommand{\FV}{\ensuremath{\mathrm{FV}}}

\newcommand{\reducesto}[0]{\ensuremath{\leadsto}}
\newcommand{\areducesto}{\ensuremath{\reducesto_{\textrm{a}}}}
\newcommand{\stepsto}[0]{\ensuremath{\longrightarrow}}
\newcommand{\Stepsto}{\Longrightarrow}
\newcommand{\EC}{\ensuremath{\mathcal{E}}}

\newcommand{\BL}{\ensuremath{\mathsf{BL}}}

\newcommand{\dom}{\ensuremath{\mathsf{dom}}}

\newcommand{\Res}{\keyw{res}}

%% Handler projections.
\newcommand{\mret}{\mathrm{ret}}
\newcommand{\mops}{\mathrm{ops}}
\newcommand{\hret}{H^{\mret}}
\newcommand{\hval}{\hret}
\newcommand{\hops}{H^{\mops}}
%\newcommand{\hex}{H^{\mathrm{ex}}}
\newcommand{\hell}{H^{\ell}}

\newcommand{\depth}{\delta}

\newcommand{\alertbox}[2]{{\par\noindent\small\color{red} \framebox{\parbox{\dimexpr\linewidth-2\fboxsep-2\fboxrule}{\textbf{#1:} #2}}}}
\newcommand{\todo}[1]{\alertbox{TODO}{#1}}
\newcommand{\dhil}[1]{\alertbox{Daniel}{#1}}

%%
%% Labels
%%
\newcommand{\slab}[1]{\ensuremath{\mathsf{#1}}}
\newcommand{\rulelabel}[2]{\ensuremath{\mathsf{#1\textrm{-}#2}}}
\newcommand{\klab}[1]{\rulelabel{K}{#1}}
\newcommand{\semlab}[1]{\rulelabel{S}{#1}}
\newcommand{\usemlab}[1]{\rulelabel{U}{#1}}
\newcommand{\tylab}[1]{\rulelabel{T}{#1}}
\newcommand{\mlab}[1]{\rulelabel{M}{#1}}
\newcommand{\siglab}[1]{\rulelabel{Sig}{#1}}
\newcommand{\rowlab}[1]{\rulelabel{R}{#1}}

%%
%% Syntactic categories.
%%
\newcommand{\CatName}[1]{\textrm{#1}}
\newcommand{\CompCat}{\CatName{Comp}}
\newcommand{\UCompCat}{\CatName{UComp}}
\newcommand{\UValCat}{\CatName{UVal}}
\newcommand{\SCompCat}{\CatName{SComp}}
\newcommand{\SValCat}{\CatName{SVal}}
\newcommand{\SPatCat}{\CatName{SPat}}
\newcommand{\ValCat}{\CatName{Val}}
\newcommand{\VarCat}{\CatName{Var}}
\newcommand{\ValTypeCat}{\CatName{VType}}
\newcommand{\CompTypeCat}{\CatName{CType}}
\newcommand{\HandlerTypeCat}{\CatName{HType}}
\newcommand{\PresenceCat}{\CatName{Presence}}
\newcommand{\TypeCat}{\CatName{Type}}
\newcommand{\TyVarCat}{\CatName{TVar}}
\newcommand{\KindCat}{\CatName{Kind}}
\newcommand{\RowCat}{\CatName{Row}}
\newcommand{\EffectCat}{\CatName{Effect}}
\newcommand{\TermCat}{\CatName{Term}}
\newcommand{\LabelCat}{\CatName{Label}}
\newcommand{\TyEnvCat}{\CatName{TyEnv}}
\newcommand{\KindEnvCat}{\CatName{KindEnv}}
\newcommand{\EvalCat}{\CatName{Cont}}
\newcommand{\UEvalCat}{\CatName{UCont}}
\newcommand{\HandlerCat}{\CatName{HDef}}

%%
%% Lindley's array stuff.
%%
\newcommand{\ba}{\begin{array}}
\newcommand{\ea}{\end{array}}

\newcommand{\bl}{\ba[t]{@{}l@{}}}
\newcommand{\el}{\ea}


%%
%% Lindley's syntax, reductions, equations, and derivation environments.
%%
\newenvironment{syntax}{\[\ba{@{}l@{\quad}r@{~}c@{~}l@{}}}{\ea\]\ignorespacesafterend}
\newenvironment{reductions}{\[\ba{@{}l@{\qquad}@{}r@{~~}c@{~~}l@{}}}{\ea\]\ignorespacesafterend}

\newenvironment{eqs}{\ba{@{}r@{~}c@{~}l@{}}}{\ea}
\newenvironment{equations}{\[\ba{@{}r@{~}c@{~}l@{}}}{\ea\]\ignorespacesafterend}
\newcommand\numberthis{\addtocounter{equation}{1}\tag{$\ast$\theequation}} % Numbering equations
\newenvironment{derivation}{\begin{displaymath}\ba{@{}r@{~}l@{}}}{\ea\end{displaymath}\ignorespacesafterend}
\newcommand{\reason}[1]{\quad (\text{#1})}


\newenvironment{smathpar}{\vspace{-3ex}\small\begin{mathpar}}{\end{mathpar}\normalsize\ignorespacesafterend}

%%
%% Lists
%%
\newcommand{\nil}{\ensuremath{[]}}
\newcommand{\cons}{\ensuremath{::}}

\newcommand{\concat}{\mathbin{+\!\!+}}
\newcommand{\revconcat}{\mathbin{\widehat{\concat}}}

%%
%% CPS notation
%%
% static / dynamic stuff
\newcommand{\scol}[1]{{\color{blue}#1}}
\newcommand{\dcol}[1]{{\color{red}#1}}

\newcommand{\static}[1]{\scol{\overline{#1}}}
\newcommand{\dynamic}[1]{\dcol{\underline{#1}}}
\newcommand{\nary}[1]{\overline{#1}}

\newcommand{\slam}{\static{\lambda}}
\newcommand{\dlam}{\dynamic{\lambda}}
\newcommand{\sapp}{\mathbin{\static{@}}}
\newcommand{\dapp}{\mathbin{\dynamic{@}}}

\newcommand{\reify}{\mathord{\downarrow}}
\newcommand{\reflect}{\mathord{\uparrow}}

\newcommand{\scons}{\mathbin{\static{\cons}}}
\newcommand{\dcons}{\mathbin{\dynamic{\cons}}}

\newcommand{\snil}{\static{\nil}}
\newcommand{\dnil}{\dynamic{\nil}}

\newcommand{\sRecord}[1]{\static{\langle}#1\static{\rangle}}
\newcommand{\dRecord}[1]{\dynamic{\langle}#1\dynamic{\rangle}}

\newcommand{\sQ}{\mathcal{Q}}
\newcommand{\sV}{\mathcal{V}}
\newcommand{\sW}{\mathcal{W}}

\newcommand{\sM}{\mathcal{M}}
\renewcommand{\snil}{\reflect \dnil}

\newcommand{\cps}[1]{\ensuremath{\llbracket #1 \rrbracket}}
\newcommand{\pcps}[1]{\top\cps{#1}}

\newcommand{\hforward}{M_{\textrm{forward}}}
\newcommand{\hid}{V_{\textrm{id}}}

% continuation application
\newcommand{\kapp}{\keyw{app}}

%%% Continuation names
%%% The following set of macros are a bit more consistent with those
%%% currently used by the abstract machine, and don't use the plural
%%% convention of functional programming.

% dynamic
\newcommand{\dlf}{f}    % let frames
\newcommand{\dlk}{fs}   % let continuations
\newcommand{\dhf}{q}    % handler frames
\newcommand{\dhk}{ks}    % handler continuations
\newcommand{\dhkr}{rs}  % reverse handler continuations
\newcommand{\dLet}{\dynamic{\Let}}
\newcommand{\dIn}{\dynamic{\In}}

% static
\newcommand{\slf}{\phi}    % let frames
\newcommand{\slk}{\sigma}  % let continuations
\newcommand{\shf}{\theta}  % handler frames
\newcommand{\shk}{\kappa}  % handler continuations
\newcommand{\sLet}{\static{\Let}}
\newcommand{\sIn}{\static{\In}}
% \newcommand{\sk}{\kappa}
% \newcommand{\sks}{\mathit{\kappa s}}
\newcommand{\sks}{\kappa}
% \newcommand{\sh}{\eta}
\newcommand{\sk}{\theta}
\newcommand{\sh}{\chi}
\newcommand{\sx}{\varepsilon}

\newcommand{\sP}{\mathcal{P}}
\newcommand{\VS}{VS}
\newcommand{\Vmap}{\keyw{vmap}}
\newcommand{\Vmapsnd}{\keyw{vmapsnd}}
\newcommand{\Fun}{\keyw{fun}}

% Canonical variables for handler components
\newcommand{\vhret}{h^{\mret}}
\newcommand{\vhops}{h^{\mops}}
\newcommand{\svhret}{\chi^{\mret}}
\newcommand{\svhops}{\chi^{\mops}}

% \newcommand{\dk}{\dRecord{fs,\dRecord{\vhret,\vhops}}}
\newcommand{\dk}{k}

%
\renewcommand{\hid}{V_{\mops}}
\newcommand{\kid}{V_\mathrm{id}}
\newcommand{\rid}{V_{\mret}}

% Examples
\newcommand{\Pipe}{\dec{pipe}}
\newcommand{\Copipe}{\dec{copipe}}
\newcommand{\Ones}{\dec{ones}}
\newcommand{\Yield}{\dec{Yield}}
\newcommand{\Await}{\dec{Await}}
\newcommand{\AddTwo}{\ensuremath{\dec{add}_2}}
\newcommand{\Option}{\dec{Option}}
\newcommand{\Some}{\dec{Some}}
\newcommand{\None}{\dec{None}}
\newcommand{\Toss}{\dec{Toss}}
\newcommand{\toss}{\dec{toss}}
\newcommand{\Heads}{\dec{Heads}}
\newcommand{\Tails}{\dec{Tails}}
\newcommand{\Exn}{\dec{Exn}}
\newcommand{\fail}{\dec{fail}}
\newcommand{\optionalise}{\dec{optionalise}}
\newcommand{\bind}{\ensuremath{\gg\!=}}
\newcommand{\return}{\dec{return}}
\newcommand{\faild}{\dec{withDefault}}

% Abstract machine
\newcommand{\cek}[1]{\ensuremath{\langle #1 \rangle}}
% Environments
\newcommand{\env}{\ensuremath{\gamma}}
% restrict an environment
\newcommand{\res}{\backslash}

% abstract machine translations
\newcommand{\val}[2]{\llbracket #1 \rrbracket #2}
\newcommand{\inv}[1]{\llparenthesis #1 \rrparenthesis}

% configurations
\newcommand{\conf}{\mathcal{C}}

% UNIX example
\newcommand{\UNIX}{UNIX}
\newcommand{\OSname}[0]{Tiny UNIX}
\newcommand{\exit}{\dec{exit}}
\newcommand{\Exit}{\dec{Exit}}
\newcommand{\Status}{\dec{Status}}
\newcommand{\status}{\dec{status}}
\newcommand{\basicIO}{\dec{basicIO}}
\newcommand{\Putc}{\dec{Putc}}
\newcommand{\putc}{\dec{putc}}
\newcommand{\UFile}{\dec{File}}
\newcommand{\UFD}{\dec{FileDescr}}
\newcommand{\fwrite}{\dec{fwrite}}
\newcommand{\iter}{\dec{iter}}
\newcommand{\stdout}{\dec{stdout}}
\newcommand{\IO}{\dec{IO}}
\newcommand{\BIO}{\dec{BIO}}
\newcommand{\Alice}{\dec{Alice}}
\newcommand{\Bob}{\dec{Bob}}
\newcommand{\Root}{\dec{Root}}
\newcommand{\User}{\dec{User}}
\newcommand{\environment}{\dec{env}}
\newcommand{\EnvE}{\dec{Session}}
\newcommand{\Ask}{\dec{Ask}}
\newcommand{\whoami}{\dec{whoami}}
\newcommand{\Su}{\dec{Su}}
\newcommand{\su}{\dec{su}}
\newcommand{\sessionmgr}{\dec{sessionmgr}}
\newcommand{\echo}{\dec{echo}}
\newcommand{\strlit}[1]{\texttt{"#1"}}
\newcommand{\nondet}{\dec{nondet}}
\newcommand{\Fork}{\dec{Fork}}
\newcommand{\fork}{\dec{fork}}
\newcommand{\Interrupt}{\dec{Interrupt}}
\newcommand{\interrupt}{\dec{interrupt}}
\newcommand{\Pstate}{\dec{Pstate}}
\newcommand{\Done}{\dec{Done}}
\newcommand{\Suspended}{\dec{Paused}}
\newcommand{\slice}{\dec{slice}}
\newcommand{\reifyP}{\dec{reifyProcess}}
\newcommand{\timeshare}{\dec{timeshare}}
\newcommand{\runNext}{\dec{runNext}}
\newcommand{\concatMap}{\dec{concatMap}}
\newcommand{\State}{\dec{State}}
\newcommand{\runState}{\dec{runState}}
\newcommand{\Uget}{\dec{get}}
\newcommand{\Uput}{\dec{put}}
\newcommand{\nl}{\textbackslash{}n}
\newcommand{\quoteRitchie}{\dec{ritchie}}
\newcommand{\quoteHamlet}{\dec{hamlet}}
\newcommand{\Proc}{\dec{Proc}}
\newcommand{\schedule}{\dec{schedule}}
\newcommand{\fsname}{BSFS}
\newcommand{\FileSystem}{\dec{FileSystem}}
\newcommand{\Directory}{\dec{Directory}}
\newcommand{\DataRegion}{\dec{DataRegion}}
\newcommand{\INode}{\dec{INode}}
\newcommand{\IList}{\dec{IList}}
\newcommand{\fileRW}{\dec{fileRW}}
\newcommand{\fileAlloc}{\dec{fileCO}}
\newcommand{\URead}{\dec{Read}}
\newcommand{\UWrite}{\dec{Write}}
\newcommand{\UCreate}{\dec{Create}}
\newcommand{\UOpen}{\dec{Open}}
\newcommand{\fread}{\dec{fread}}
\newcommand{\fcreate}{\dec{fcreate}}
\newcommand{\Ucreate}{\dec{create}}
\newcommand{\redirect}{\texttt{>}}
\newcommand{\fopen}{\dec{fopen}}
\newcommand{\lookup}{\dec{lookup}}
\newcommand{\modify}{\dec{modify}}
\newcommand{\fileIO}{\dec{fileIO}}
\newcommand{\ULink}{\dec{Link}}
\newcommand{\UUnlink}{\dec{Unlink}}
\newcommand{\flink}{\dec{flink}}
\newcommand{\funlink}{\dec{funlink}}
\newcommand{\remove}{\dec{remove}}
\newcommand{\FileLU}{\dec{FileLU}}
\newcommand{\fileLU}{\dec{fileLU}}

%%
%% Some control operators
%%
\newcommand{\Cupto}{\keyw{cupto}}
\newcommand{\Set}{\keyw{set}}
\newcommand{\newPrompt}{\keyw{newPrompt}}
\newcommand{\Callcc}{\keyw{callcc}}
\newcommand{\Callcomc}{\ensuremath{\keyw{callcomp}}}
\newcommand{\textCallcomc}{callcomp}
\newcommand{\Throw}{\keyw{throw}}
\newcommand{\Continue}{\keyw{resume}}
\newcommand{\Catch}{\keyw{catch}}
\newcommand{\Catchcont}{\keyw{catchcont}}
\newcommand{\Control}{\keyw{control}}
\newcommand{\Prompt}{\#}
\newcommand{\Controlz}{\keyw{control0}}
\newcommand{\Promptz}{\#_0}
\newcommand{\Escape}{\keyw{escape}}
\newcommand{\shift}{\keyw{shift}}
\newcommand{\shiftz}{\keyw{shift0}}
\def\sigh#1{%
\pmb{\left\langle\vphantom{#1}\right.}%
#1%
\pmb{\left.\vphantom{#1}\right\rangle}}
\newcommand{\llambda}{\ensuremath{\pmb{\lambda}}}
\newcommand{\reset}[1]{\pmb{\langle} #1 \pmb{\rangle}}
\newcommand{\resetz}[1]{\pmb{\langle} #1 \pmb{\rangle}_0}
\newcommand{\fcontrol}{\keyw{fcontrol}}
\newcommand{\fprompt}{\%}
\newcommand{\splitter}{\keyw{splitter}}
\newcommand{\abort}{\keyw{abort}}
\newcommand{\calldc}{\keyw{calldc}}
\newcommand{\J}{\keyw{J}}
\newcommand{\JI}{\keyw{J}\,\keyw{I}}
\newcommand{\FelleisenC}{\ensuremath{\keyw{C}}}
\newcommand{\FelleisenF}{\ensuremath{\keyw{F}}}
\newcommand{\cont}{\keyw{cont}}
\newcommand{\Cont}{\dec{Cont}}
\newcommand{\Algol}{Algol~60}
\newcommand{\qq}[1]{\ensuremath{\ulcorner #1 \urcorner}}
\newcommand{\prompttype}{\dec{Prompt}}

% Language macros
\newcommand{\Frank}{Frank}
\newcommand{\SML}{SML}
\newcommand{\SMLNJ}{\SML{}/NJ}
\newcommand{\OCaml}{OCaml}


% Tikz figures
\newcommand\toggle{
\begin{tikzpicture}[->,>=stealth',level/.style={sibling distance = 5cm/##1,
  level distance = 1.5cm}]
\node [opnode] {$\Get~\Unit$}
    child{ node [opnode] {$\Put~\False$}
           child{ node [leaf] {$\True$}
                  edge from parent node[left] {$\Unit$}}
           edge from parent node[above left] {$\True$}
    }
    child{ node [opnode] {$\Put~v$}
           child{ node [leaf] {$\False$}
                  edge from parent node[right] {$\res{()}$}}
           edge from parent node[above right] {$\res{\False}$}
    }
;
\end{tikzpicture}}

%% Information about the title, etc.
% \title{Higher-Order Theories of Handlers for Algebraic Effects}
% \title{Handlers for Algebraic Effects: Applications, Compilation, and Expressiveness}
% \title{Applications, Compilation, and Expressiveness for Effect Handlers}
% \title{Handling Computational Effects}
% \title{Programming Computable Effectful Functions}
% \title{Handling Effectful Computations}
\title{Foundations for Programming and Implementing Effect Handlers}
\author{Daniel Hillerström}

%% If the year of submission is not the current year, uncomment this line and
%% specify it here:
\submityear{2020}

%% Specify the abstract here.
\abstract{%
  An abstract\dots
}

%% Now we start with the actual document.
\begin{document}
\raggedbottom
%% First, the preliminary pages
\begin{preliminary}

%% This creates the title page
\maketitle

%% Acknowledgements
\begin{acknowledgements}
  List of people to thank
  \begin{itemize}
    \item Sam Lindley
    \item John Longley
    \item Christophe Dubach
    \item KC Sivaramakrishnan
    \item Stephen Dolan
    \item Anil Madhavapeddy
    \item Gemma Gordon
    \item Leo White
    \item Andreas Rossberg
    \item Robert Atkey
    \item Jeremy Yallop
    \item Simon Fowler
    \item Craig McLaughlin
    \item Garrett Morris
    \item James McKinna
    \item Brian Campbell
    \item Paul Piho
    \item Amna Shahab
    \item Gordon Plotkin
    \item Ohad Kammar
    \item School of Informatics (funding)
    \item Google (Kevin Millikin, Dmitry Stefantsov)
    \item Microsoft Research (Daan Leijen)
  \end{itemize}
\end{acknowledgements}

%% Next we need to have the declaration.
% \standarddeclaration
\begin{declaration}
  I declare that this thesis was composed by myself, that the work
  contained herein is my own except where explicitly stated otherwise
  in the text, and that this work has not been submitted for any other
  degree or professional qualification except as specified.
\end{declaration}

%% Finally, a dedication (this is optional -- uncomment the following line if
%% you want one).
% \dedication{To my mummy.}
\dedication{\emph{To be or to do}}

% \begin{preface}
% A preface will possibly appear here\dots
% \end{preface}

%% Create the table of contents
\setcounter{secnumdepth}{2} % Numbering on sections and subsections
\setcounter{tocdepth}{1} % Show chapters, sections and subsections in TOC
%\singlespace
\tableofcontents
%\doublespace

%% If you want a list of figures or tables, uncomment the appropriate line(s)
% \listoffigures
% \listoftables
\end{preliminary}

%%%%%%%%%%%%%%%%%%%%%%%%%%%%%%%%%%%
%%          Main content         %%
%%%%%%%%%%%%%%%%%%%%%%%%%%%%%%%%%%%

%%
%% Introduction
%%
\chapter{Introduction}
\label{ch:introduction}
An enthralling introduction\dots
%
Motivation: 1) compiler perspective: unifying control abstraction,
lean runtime, desugaring of async/await, generators/iterators, 2)
giving control to programmers, safer microkernels, everything as a
library.

\section{Thesis outline}
Thesis outline\dots

\section{Typographical conventions}
Explain conventions\dots

\part{Background}
\label{p:background}

\chapter{The state of effectful programming}
\label{ch:related-work}

\section{Type and effect systems}
\section{Monadic programming}

\chapter{Continuations}
\label{ch:continuations}

\section{Zoo of control operators}
Describe how effect handlers fit amongst shift/reset, prompt/control,
callcc, J, catchcont, etc.

\section{Implementation strategies}


\part{Design}

\chapter{A ML-flavoured programming language}
\label{ch:base-language}

In this chapter we introduce a core calculus, \BCalc{}, which we shall
later use as the basis for exploration of design considerations for
effect handlers. This calculus is based on \CoreLinks{} by
\citet{LindleyC12}, which distils the essence of the functional
multi-tier web-programming language
\Links{}~\cite{CooperLWY06}. \Links{} belongs to the
ML-family~\cite{MilnerTHM97} of programming languages as it features
typical characteristics of ML languages such as a static type system
supporting parametric polymorphism with type inference support (in
fact Links supports first-class polymorphism), and its evaluation
semantics is strict. However, \Links{} differentiates itself from the
rest of the ML-family by making crucial use of \emph{row polymorphism}
to support extensible records, variants, and tracking of computational
effects. Thus \Links{} has a rather strong emphasis on structural
types rather than nominal types.

\CoreLinks{} captures all of these properties of \Links{}. Our
calculus \BCalc{} differs in several aspects from \CoreLinks{}. For
example, the underlying formalism of \CoreLinks{} is call-by-value,
whilst the formalism of \BCalc{} is \emph{fine-grain
  call-by-value}~\cite{LevyPT03}, which shares similarities with
A-normal form (ANF)~\cite{FlanaganSDF93} as it syntactically
distinguishes between value and computation terms by mandating every
intermediate computation being named. However unlike ANF, fine-grain
call-by-value remains closed under $\beta$-reduction. The reason for
choosing fine-grain call-by-value as our formalism is entirely due to
convenience. As we shall see in Chapter~\ref{ch:unary-handlers}
fine-grain call-by-value is a convenient formalism for working with
continuations. Another point of difference between \CoreLinks{} and
\BCalc{} is that the former models the integrated database query
sublanguage of \Links{}. We do not consider the query sublanguage at
all, and instead our focus is entirely on modelling the interaction
and programming with computational effects.

\section{Syntax and static semantics}
\label{sec:syntax-base-language}

In \BCalc{}, types are precursory to terms, as it is intrinsically
typed. Thus we begin by presenting the syntax of its kind and type
structure in Section~\ref{sec:base-language-types}. Subsequently in
Section~\ref{sec:base-language-terms} we present the term syntax,
before presenting the formation rules for types in
Section~\ref{sec:base-language-type-rules}.

% Typically the presentation of a programming language begins with its
% syntax. If the language is typed there are two possible starting
% points: Either one presents the term syntax first, or alternatively,
% the type syntax first. Although the choice may seem rather benign
% there is, however, a philosophical distinction to be drawn between
% them. Terms are, on their own, entirely meaningless, whilst types
% provide, on their own, an initial approximation of the semantics of
% terms. This is particularly true in an intrinsic typed system perhaps
% less so in an extrinsic typed system. In an intrinsic system types
% must necessarily be precursory to terms, as terms ultimately depend on
% the types. Following this argument leaves us with no choice but to
% first present the type syntax of \BCalc{} and subsequently its term
% syntax.

\subsection{Types and their kinds}
\label{sec:base-language-types}
%
\begin{figure}
  \begin{syntax}
    \slab{Value types}    &A,B  &::= & A \to C
                                 \mid \alpha
                                 \mid \forall \alpha^K.C
                                 \mid \Record{R}
                                 \mid [R]\\
\slab{Computation types}
                      &C,D  &::= & A \eff E \\
\slab{Effect types}   &E    &::= & \{R\}\\
\slab{Row types}      &R    &::= & \ell : P;R \mid \rho \mid \cdot \\
\slab{Presence types} &P    &::= & \Pre{A} \mid \Abs \mid \theta\\
%\slab{Labels}         &\ell &    &                \\
% \slab{Types}          &T    &::= & A \mid C \mid E \mid R \mid P \\
\slab{Kinds}          &K    &::= & \Type \mid \Row_\mathcal{L} \mid \Presence
                             \mid \Comp \mid \Effect \\
\slab{Label sets}     &\mathcal{L} &::=& \emptyset \mid \{\ell\} \uplus \mathcal{L}\\
%\slab{Type variables} &\alpha, \rho, \theta& \\
\slab{Type environments} &\Gamma &::=& \cdot \mid \Gamma, x:A \\
\slab{Kind environments} &\Delta &::=& \cdot \mid \Delta, \alpha:K
  \end{syntax}
  \caption{Syntax of types, kinds, and their environments.}
  \label{fig:base-language-types}
\end{figure}
%
Figure~\ref{fig:base-language-types} depicts the syntax for types,
kinds, and their environments.
%
\paragraph{Value types}
We distinguish between values and computations at the level of
types. Value types comprise the function type $A \to C$, which maps
values of type $A$ to computations of type $C$; the polymorphic type
$\forall \alpha^K . C$ is parameterised by a type variable $\alpha$ of
kind $K$; and the record type $\Record{R}$ represents records with
fields constrained by row $R$. Dually, the variant type $[R]$
represents tagged sums constrained by row $R$.

\paragraph{Computation types and effect types}
The computation type $A \eff E$ is given by a value type $A$ and an
effect type $E$, which specifies the effectful operations a
computation inhabiting this type may perform. An effect type
$E = \{R\}$ is constrained by row $R$.

\paragraph{Row types}
Row types play a pivotal role in our type system as effect, record,
and variant types are uniformly given by row types. A \emph{row type}
describes a collection of distinct labels, each annotated by a
presence type. A presence type indicates whether a label is
\emph{present} with type $A$ ($\Pre{A}$), \emph{absent} ($\Abs$) or
\emph{polymorphic} in its presence ($\theta$).
%
Row types are either \emph{closed} or \emph{open}. A closed row type
ends in~$\cdot$, whilst an open row type ends with a \emph{row
  variable} $\rho$ (in an effect row we usually use $\varepsilon$
rather than $\rho$ and refer to it as an \emph{effect variable}).
%
The row variable in an open row type can be instantiated with
additional labels. Each label may occur at most once in each row (we
enforce this restriction at the level of kinds). We identify rows up
to the reordering of labels. We assume structural equality on labels.
%
% \begin{mathpar}
%   \inferrule*[Lab=\rowlab{Closed}]
%     {~}
%     {\cdot \equiv_{\mathrm{row}} \cdot}

%   \inferrule*[Lab=\rowlab{Open}]
%     {~}
%     {\rho \equiv_{\mathrm{row}} \rho}

%   \inferrule*[Lab=\rowlab{Head}]
%     {R \equiv_{\mathrm{row}} R'}
%     {\ell:P;R \equiv_{\mathrm{row}} \ell:P;R'}

%   \inferrule*[Lab=\rowlab{Swap}]
%     {R \equiv_{\mathrm{row}} R'}
%     {\ell:P;\ell':P';R \equiv_{\mathrm{row}} \ell':P';\ell:P;R'}
% \end{mathpar}
% %
% The last rule $\rowlab{Swap}$ let us identify rows up to the
% reordering of labels.  For instance, the two rows
% $\ell_1 : P_1; \cdots; \ell_n : P_n; \cdot$ and
% $\ell_n : P_n; \cdots ; \ell_1 : P_1; \cdot$ are equivalent.
%
Absent labels in closed rows are redundant.
%
The standard zero and unit types are definable using rows. We define
the zero type as the empty, closed variant $\ZeroType \defas
[\cdot]$. Dually, the unit type is defined as the empty, closed record
type, i.e. $\UnitType \defas \Record{\cdot}$. We shall often omit
$\cdot$ for closed rows.

%
\begin{figure}
\begin{mathpar}
% alpha : K
  \inferrule*[Lab=\klab{TyVar}]
    { }
    {\Delta, \alpha : K \vdash \alpha : K}

% Computation
    \inferrule*[Lab=\klab{Comp}]
    { \Delta \vdash A : \Type \\
      \Delta \vdash E : \Effect \\
    }
    {\Delta \vdash A \eff E : \Comp}

% A -E-> B, A : Type, E : Row, B : Type
  \inferrule*[Lab=\klab{Fun}]
    { \Delta \vdash A : \Type \\
      \Delta \vdash C : \Comp  \\
    }
    {\Delta \vdash A \to C : \Type}

% forall alpha : K . A : Type
  \inferrule*[Lab=\klab{Forall}]
    { \Delta, \alpha : K \vdash C : \Comp}
    {\Delta \vdash \forall \alpha^K . \, C : \Type}

% Record
  \inferrule*[Lab=\klab{Record}]
    { \Delta \vdash R : \Row_\emptyset}
    {\Delta \vdash \Record{R} : \Type}

% Variant
  \inferrule*[Lab=\klab{Variant}]
    { \Delta \vdash R : \Row_\emptyset}
    {\Delta \vdash [R] : \Type}

% Effect
  \inferrule*[Lab=\klab{Effect}]
    { \Delta \vdash R : \Row_\emptyset}
    {\Delta \vdash \{R\} : \Effect}

% Present
  \inferrule*[Lab=\klab{Present}]
    {\Delta \vdash A : \Type}
    {\Delta \vdash \Pre{A} : \Presence}

% Absent
  \inferrule*[Lab=\klab{Absent}]
    { }
    {\Delta \vdash \Abs : \Presence}

% Empty row
  \inferrule*[Lab=\klab{EmptyRow}]
    { }
    {\Delta \vdash \cdot : \Row_\mathcal{L}}

% Extend row
  \inferrule*[Lab=\klab{ExtendRow}]
    { \Delta \vdash P : \Presence \\
      \Delta \vdash R : \Row_{\mathcal{L} \uplus \{\ell\}}
    }
    {\Delta \vdash \ell : P;R : \Row_\mathcal{L}}
\end{mathpar}
\caption{Kinding rules}
\label{fig:base-language-kinding}
\end{figure}
%

\paragraph{Kinds}
The kinds comprise $\Type$ for regular type variables, $\Presence$ for
presence variables, $\Comp$ for computation type variables, $\Effect$
for effect variables, and lastly $\Row_{\mathcal{L}}$ for row
variables.
%
The formation rules for kinds are given in
Figure~\ref{fig:base-language-kinding}. The kinding judgement
$\Delta \vdash T : K$ states that type $T$ has kind $K$ in kind
environment $\Delta$.
%
The row kind is annotated by a set of labels $\mathcal{L}$. We use
this set to track the labels of a given row type to ensure uniqueness
amongst labels in each row type. For example, the kinding rule
$\klab{ExtendRow}$ uses this set to constrain which labels may be
mentioned in the tail of $R$. We shall elaborate on this in
Section~\ref{sec:row-polymorphism}.

\paragraph{Environments}
Kind and type environments are right-extended sequences of bindings. A
kind environment binds type variables to their kinds, whilst a type
environment binds term variables to their types.

% Value types comprise the function type $A \to C$, whose domain
% is a value type and its codomain is a computation type $B \eff E$,
% where $E$ is an effect type detailing which effects the implementing
% function may perform. Value types further comprise type variables
% $\alpha$ and quantification $\forall \alpha^K.C$, where the quantified
% type variable $\alpha$ is annotated with its kind $K$. Finally, the
% value types also contains record types $\Record{R}$ and variant types
% $[R]$, which are built up using row types $R$. An effect type $E$ is
% also built up using a row type. A row type is a sequence of fields of
% labels $\ell$ annotated with their presence information $P$. The
% presence information denotes whether a label is present $\Pre{A}$ with
% some type $A$, absent $\Abs$, or polymorphic in its presence
% $\theta$. A row type may be either \emph{open} or \emph{closed}. An
% open row ends in a row variable $\rho$ which can be instantiated with
% additional fields, effectively growing the row, whilst a closed row
% ends in $\cdot$, meaning the row cannot grow further.

% The kinds comprise $\Type$ for regular type variables, $\Presence$ for
% presence variables, $\Comp$ for computation type variables, $\Effect$
% for effect variables, and lastly $\Row_{\mathcal{L}}$ for row
% variables. The row kind is annotated by a set of labels
% $\mathcal{L}$. We use this set to track the labels of a given row type
% to ensure uniqueness amongst labels in each row type. We shall
% elaborate on this in Section~\ref{sec:row-polymorphism}.

\subsection{Terms}
\label{sec:base-language-terms}
%
\begin{figure}
\begin{syntax}
\slab{Variables}     &x \in \VarCat&&\\
\slab{Values}        &V,W \in \ValCat  &::= & x
                                        \mid \lambda x^A .\, M \mid \Lambda \alpha^K .\, M
                                        \mid \Record{} \mid \Record{\ell = V;W} \mid (\ell~V)^R \\
                     &     &    &\\
\slab{Computations}  &M,N \in \CompCat  &::= & V\,W \mid V\,A\\
                     &                  &\mid& \Let\; \Record{\ell=x;y} = V \; \In \; N\\
                     &                  &\mid& \Case\; V \{\ell~x \mapsto M; y \mapsto N\} \mid \Absurd^C~V\\
                     &                  &\mid& \Return~V \mid \Let \; x \revto M \; \In \; N
\end{syntax}

\caption{Term syntax of \BCalc{}.}
\label{fig:base-language-term-syntax}
\end{figure}
%
The syntax for terms is given in
Figure~\ref{fig:base-language-term-syntax}. We assume countably
infinite set of names $\VarCat$ from which we draw fresh variable
names at will. We shall typically denote term variables by $x$, $y$,
or $z$.
%
The syntax partitions terms into values and computations.
%
Value terms comprise variables ($x$), lambda abstraction
($\lambda x^A . \, M$), type abstraction ($\Lambda \alpha^K . \, M$),
and the introduction forms for records and variants. Records are
introduced using the empty record $\Record{}$ and record extension
$\Record{\ell = V; W}$, whilst variants are introduced using injection
$(\ell~V)^R$, which injects a field with label $\ell$ and value $V$
into a row whose type is $R$. We include the row type annotation in
order to support bottom-up type reconstruction.

All elimination forms are computation terms. Abstraction and type
abstraction are eliminated using application ($V\,W$) and type
application ($V\,A$) respectively.
%
The record eliminator $(\Let \; \Record{\ell=x;y} = V \; \In \; N)$
splits a record $V$ into $x$, the value associated with $\ell$, and
$y$, the rest of the record. Non-empty variants are eliminated using
the case construct ($\Case\; V\; \{\ell~x \mapsto M; y \mapsto N\}$),
which evaluates the computation $M$ if the tag of $V$ matches
$\ell$. Otherwise it falls through to $y$ and evaluates $N$.  The
elimination form for empty variants is ($\Absurd^C~V$).
%
There is one computation introduction form, namely, the trivial
computation $(\Return~V)$ which returns value $V$. Its elimination
form is the expression $(\Let \; x \revto M \; \In \; N)$ which evaluates
$M$ and binds the result value to $x$ in $N$.
%

%
As our calculus is intrinsically typed, we annotate terms with type or
kind information (term abstraction, type abstraction, injection,
operations, and empty cases). However, we shall omit these annotations
whenever they are clear from context.

\subsection{Typing rules}
\label{sec:base-language-type-rules}
%
\begin{figure}
Values
\begin{mathpar}
% Variable
  \inferrule*[Lab=\tylab{Var}]
    {x : A \in \Gamma}
    {\typv{\Delta;\Gamma}{x : A}}

% Abstraction
  \inferrule*[Lab=\tylab{Lam}]
    {\typ{\Delta;\Gamma, x : A}{M : C}}
    {\typv{\Delta;\Gamma}{\lambda x^A .\, M : A \to C}}

% Polymorphic abstraction
  \inferrule*[Lab=\tylab{PolyLam}]
    {\typv{\Delta,\alpha : K;\Gamma}{M : C} \\
     \alpha \notin FTV(\Gamma)
    }
    {\typv{\Delta;\Gamma}{\Lambda \alpha^K .\, M : \forall \alpha^K . \,C}}
\\
% unit : ()
  \inferrule*[Lab=\tylab{Unit}]
    { }
    {\typv{\Delta;\Gamma}{\Record{} : \Record{}}}

% Extension
  \inferrule*[Lab=\tylab{Extend}]
    { \typv{\Delta;\Gamma}{V : A} \\
      \typv{\Delta;\Gamma}{W : \Record{\ell:\Abs;R}}
    }
    {\typv{\Delta;\Gamma}{\Record{\ell=V;W} : \Record{\ell:\Pre{A};R}}}

% Inject
  \inferrule*[Lab=\tylab{Inject}]
    {\typv{\Delta;\Gamma}{V : A}}
    {\typv{\Delta;\Gamma}{(\ell~V)^R : [\ell : \Pre{A}; R]}}
\end{mathpar}
Computations
\begin{mathpar}
% Application
  \inferrule*[Lab=\tylab{App}]
    {\typv{\Delta;\Gamma}{V : A \to C} \\
     \typv{\Delta;\Gamma}{W : A}
    }
    {\typ{\Delta;\Gamma}{V\,W : C}}

% Polymorphic application
  \inferrule*[Lab=\tylab{PolyApp}]
    {\typv{\Delta;\Gamma}{V : \forall \alpha^K . \, C} \\
     \Delta \vdash A : K
    }
    {\typ{\Delta;\Gamma}{V\,A : C[A/\alpha]}}

% Split
  \inferrule*[Lab=\tylab{Split}]
    {\typv{\Delta;\Gamma}{V : \Record{\ell : \Pre{A};R}} \\\\
     \typ{\Delta;\Gamma, x : A, y : \Record{\ell : \Abs; R}}{N : C}
    }
    {\typ{\Delta;\Gamma}{\Let \; \Record{\ell =x;y} = V\; \In \; N : C}}

% Case
  \inferrule*[Lab=\tylab{Case}]
    { \typv{\Delta;\Gamma}{V : [\ell : \Pre{A};R]}  \\\\
      \typ{\Delta;\Gamma,x:A}{M : C} \\\\
      \typ{\Delta;\Gamma,y:[\ell : \Abs;R]}{N : C}
    }
    {\typ{\Delta;\Gamma}{\Case \; V \{\ell\; x \mapsto M;y \mapsto N \} : C}}

% Absurd
  \inferrule*[Lab=\tylab{Absurd}]
    {\typv{\Delta;\Gamma}{V : []}}
    {\typ{\Delta;\Gamma}{\Absurd^C \; V : C}}

% Return
  \inferrule*[Lab=\tylab{Return}]
    {\typv{\Delta;\Gamma}{V : A}}
    {\typc{\Delta;\Gamma}{\Return \; V : A}{E}}
\\
% Let
  \inferrule*[Lab=\tylab{Let}]
    {\typc{\Delta;\Gamma}{M : A}{E} \\
     \typ{\Delta;\Gamma, x : A}{N : C}
    }
    {\typ{\Delta;\Gamma}{\Let \; x \revto M\; \In \; N : C}}
\end{mathpar}
\caption{Typing rules}
\label{fig:base-language-type-rules}
\end{figure}
%
\dhil{There is some rendering issue with T-labels in the typing rules.}
%
The typing rules are given by
Figure~\ref{fig:base-language-type-rules}. In each typing rule, we
implicitly assume that each type is well-kinded with respect to the
kinding environment $\Delta$.

\paragraph{Typing values} The \tylab{Var} rule states that a variable $x$ has
type $A$ if $x$ is bound to $A$ in the type environment $\Gamma$. The
\tylab{Lam} rules states that a lambda value $(\lambda x.M)$ has type
$A \to C$ if the computation $M$ has type $C$ assuming $x : A$. In the
\tylab{PolyLam} rule we make use of the set of \emph{free type
  variables} (FTV).
%
We define FTV by mutual induction over type environments, $\Gamma$,
and the type structure:
%
\[
 \ba[t]{@{~}l@{~~~~~~}c@{~}l}
\begin{eqs}
  \FTV(\alpha)   &\defas& \{\alpha\}\\
  \FTV(\forall \alpha^K.C) &\defas& \FTV(C) \setminus \{\alpha\}\\
  \FTV(A \to C)  &\defas& \FTV(A) \cup \FTV(C)\\
  \FTV(A \eff E) &\defas& \FTV(A) \cup \FTV(E)\\
  % \FTV(\{R\})    &\defas& \FTV(R)\\
  % \FTV(\Record{R}) &\defas& \FTV(R)\\
  % \FTV([R])      &\defas& \FTV(R)\\
  % \FTV(l:P;R)    &\defas& \FTV(P) \cup \FTV(R)\\
  % \FTV(\Pre{A})  &\defas& \FTV(A)\\
  % \FTV(\Abs)     &\defas& \emptyset\\
  % \FTV(\theta)   &\defas& \{\theta\}
\end{eqs} & &
\begin{eqs}
  \FTV([R])      &\defas& \FTV(R)\\
  \FTV(\Record{R}) &\defas& \FTV(R)\\
  \FTV(\{R\})    &\defas& \FTV(R)\\
  \FTV(l:P;R)    &\defas& \FTV(P) \cup \FTV(R)\\
\end{eqs}\\\\
\begin{eqs}
  \FTV(\theta)   &\defas& \{\theta\}\\
  \FTV(\Abs)     &\defas& \emptyset\\
  \FTV(\Pre{A})  &\defas& \FTV(A)
\end{eqs} & &
\begin{eqs}
  \FTV(\cdot)        &\defas& \emptyset\\
  \FTV(\Gamma,x : A) &\defas& \FTV(\Gamma) \cup \FTV(A)
\end{eqs}
\ea
\]
%
Thus the rule states that a type abstraction $(\Lambda \alpha. M)$ has
type $\forall \alpha.C$ if the computation $M$ has type $C$ assuming
$\alpha : K$ and $\alpha$ does not appear in the free type variables
of current type environment $\Gamma$. The \tylab{Unit} rule provides
the basis for all records as it simply states that the empty record
has type unit. The \tylab{Extend} rule handles record
extension. Supposing we wish to extend some record $\Record{W}$ with
$\ell = V$, that is $\Record{\ell = V; W}$. This extension has type
$\Record{\ell : \Pre{A};R}$ if and only if $V$ is well-typed and we
can ascribe $W : \Record{\ell : \Abs; R}$. Since
$\Record{\ell : \Abs; R}$ must be well-kinded with respect to
$\Delta$, the label $\ell$ cannot be mentioned in $W$, thus $\ell$
cannot occur more than once in the record. Similarly, the dual rule
\tylab{Inject} states that the injection $(\ell~V)^R$ has type
$[\ell : \Pre{A}; R]$ if the payload $V$ is well-typed. Again since
$[\ell : \Pre{A}; R]$ is well-kinded, it must be that $\ell$ is not
mentioned by $R$. In other words, the tag cannot be injected twice.

\paragraph{Typing computations}
The \tylab{App} rule states that an application $V\,W$ has computation
type $C$ if the abstractor term $V$ has type $A \to C$ and the
argument term $W$ has type $A$, that is both the argument type and the
domain type of the abstractor agree.
%
The type application rule \tylab{PolyApp} tells us that a type
application $V\,A$ is well-typed whenever the abstractor term $V$ has
the polymorphic type $\forall \alpha^K.C$ and the type $A$ has kind
$K$. This rule makes use of type substitution. We write $C[A/\alpha]$
to mean substitute some type $A$ for some type variable $\alpha$ in
some type $C$. We define type substitution as a ternary function defined as follows
%
\[
  \begin{eqs}
    (A \to C)[B/\alpha] &\defas& A[B/\alpha] \to C[B/\alpha]\\
    \alpha[B/\beta]     &\defas& \begin{cases}
                                    B & \text{if } \alpha = \beta\\
                                    \alpha & \text{otherwise}
                                  \end{cases}
  \end{eqs}
\]
\dhil{Complete the implementation\dots}
%
The \tylab{Split} rule handles typing of record destructing. When
splitting a record term $V$ on some label $\ell$ binding it to $x$ and
the remainder to $y$. The label we wish to split on must be present
with some type $A$, hence we require that
$V : \Record{\ell : \Pre{A}; R}$. This restriction prohibits us for
splitting on an absent or presence polymorphic label.  The
continuation of the splitting, $N$, must then have some computation
type $C$ subject to the following restriction: $N : C$ must be
well-typed under the additional assumptions $x : A$ and
$y : \Record{\ell : \Abs; R}$, statically ensuring that it is not
possible to split on $\ell$ again in the continuation $N$.
%
The \tylab{Case} rule is similar, but has two possible continuations:
the success continuation, $M$, and the fall-through continuation, $N$.
The label being matched must be present with some type $A$ in the type
of the scrutinee, thus we require $V : [\ell : \Pre{A};R]$. The
success continuation has some computation $C$ under the assumption
that the binder $x : A$, whilst the fall-through continuation also has
type $C$ it is subject to the restriction that the binder
$y : [\ell : \Abs;R]$ which statically enforces that no subsequent
case split in $N$ can match on $\ell$.
%
The \tylab{Absurd} states that we can ascribe any computation type to
the term $\Absurd~V$ if $V$ has the empty type $[]$.
%
The trivial computation term is typed by the \tylab{Return} rule,
which says that $\Return\;V$ has computation type $A \eff E$ if the
value $V$ has type $A$.
%
The \tylab{Let} rule types let bindings. The computation being bound,
$M$, must have computation type $A \eff E$, whilst the continuation,
$N$, must have computation $C$ subject to the additional assumption
that the binder $x : A$.

\section{Dynamic semantics}
\label{sec:base-language-dynamic-semantics}
%
\begin{figure}
\begin{reductions}
\semlab{App}   & (\lambda x^A . \, M) V &\reducesto& M[V/x] \\
\semlab{TyApp} & (\Lambda \alpha^K . \, M) A &\reducesto& M[A/\alpha] \\
\semlab{Split} & \Let \; \Record{\ell = x;y} = \Record{\ell = V;W} \; \In \; N &\reducesto& N[V/x,W/y] \\
\semlab{Case$_1$} &
  \Case \; (\ell~V)^R \{ \ell \; x \mapsto M; y \mapsto N\} &\reducesto& M[V/x] \\
\semlab{Case$_2$} &
  \Case \; (\ell~V)^R \{ \ell' \; x \mapsto M; y \mapsto N\} &\reducesto& N[(\ell~V)^R/y], \hfill\quad \text{if } \ell \neq \ell' \\
\semlab{Let} &
  \Let \; x \revto \Return \; V \; \In \; N &\reducesto& N[V/x] \\
\semlab{Lift} &
  \EC[M] &\reducesto& \EC[N], \hfill\quad \text{if } M \reducesto N \\
\end{reductions}
\begin{syntax}
\slab{Evaluation contexts} &  \mathcal{E} &::=& [\,] \mid \Let \; x \revto \mathcal{E} \; \In \; N
\end{syntax}
%%\[
% Evaluation context lift
%% \inferrule*[Lab=\semlab{Lift}]
%%   { M \reducesto N }
%%   { \mathcal{E}[M] \reducesto \mathcal{E}[N]}
%% \]

\caption{Contextual small-step semantics}
\label{fig:base-language-small-step}
\end{figure}
%
In this section we present the dynamic semantics of \BCalc{}. We
choose to use a contextual small-step semantics, since in conjunction
with fine-grain call-by-value, it yields a considerably simpler
semantics than the traditional structural operational semantics
(SOS)~\cite{Plotkin04a}, because only the rule for let bindings admits
a continuation whereas in ordinary call-by-value SOS each congruence
rule admits a continuation.
%

The semantics are based on a substitution model of computation. Thus,
before presenting the reduction rules, we define an adequacy
substitution function.
%
\paragraph{Term substitution}
We write $M[V/x]$ for the substitution of some value $V$ for some
variable $x$ in some computation term $M$. We use the same notation
for substitution on values, i.e. $W[V/x]$ denotes the substitution of
$V$ for $x$ in some value $W$. We define substitution as a ternary
function, whose signature is given by
%
\[
\cdot[\cdot/\cdot] : (\CompCat + \ValCat) \times \ValCat \times \VarCat \to \CompCat,
\]
%
and we realise it by pattern matching on the first argument.
%
\[
  \begin{eqs}
    x[V/y]               &\defas& \begin{cases}
                                     V & \text{if } x = y\\
                                     x & \text{otherwise }
                                   \end{cases}\\
    (\lambda x^A.M)[V/y] &\defas& \begin{cases}
                                    \lambda x^A.M & \text{if } x = y\\
                                    \lambda x^A.M[V/y] & \text{otherwise}
                                  \end{cases}\\
    (\Lambda \alpha^K. M)[V/y] &\defas& \Lambda \alpha^K. M[V/y]\\
    \Unit[V/y]           &\defas& \Unit\\
    \Record{\ell = W; W'}[V/y] &\defas& \Record{\ell = W[V/y]; W'[V/y]}\\
    (\ell~W)^R[V/y]      &\defas& (\ell~W[V/y])^R\\
    (W\,W')[V/y]         &\defas& W[V/y]\,W'[V/y]\\
    (W\,A)[V/y]          &\defas& W[V/y]~A\\
    (\Let\;\Record{\ell = x; y} = W\;\In\;N)[V/y] &\defas& \Let\;\Record{\ell = x; y} = W[V/y] \;\In\;N[V/y]\\
    (\Case\;(\ell~V)^R\{\ba[t]{@{}l} \ell~x \mapsto M\\
                                    ; y \mapsto N \})[V/z]\ea
                                &\defas& \begin{cases}
      \Case\;(\ell~V)^R\{\ell~x \mapsto M; y \mapsto N \} & \text{if } x = y = z\\
      \Case\;(\ell~V)^R\{\ba[t]{@{}l} \ell~x \mapsto M\\
                                      ; y \mapsto N[V/z] \}\ea & \text{if } x = z \text{ and } y \neq z\\
      \Case\;(\ell~V)^R\{ \ba[t]{@{}l}\ell~x \mapsto M[V/z]\\
                                      ; y \mapsto N \}\ea & \text{if } x \neq z \text{ and } y = z\\
      \Case\;(\ell~V)^R\{ \ba[t]{@{}l} \ell~x \mapsto M[V/z]\\
                                       ; y \mapsto N[V/z] \}\ea & \text{otherwise}
                                         \end{cases}\\
    (\Let\;x \revto M \;\In\;N)[V/y] &\defas& \begin{cases}
                                                 \Let\;x \revto M[V/y]\;\In\;N & \text{if } x = y\\
                                                 \Let\;x \revto M[V/y]\;\In\;N[V/y] & \text{otherwise}
                                              \end{cases}
  \end{eqs}
\]
%

\paragraph{Reduction semantics}
Figure~\ref{fig:base-language-small-step} depicts the reduction
rules. The application rules \semlab{App} and \semlab{TyApp}
eliminates a lambda and type abstraction, respectively, by
substituting the argument for the parameter in their body computation
$M$.
%
Record splitting is handled by the \semlab{Split} rule: splitting on
some label $\ell$ binds the payload $V$ to $x$ and the remainder $W$
to $y$ in the continuation $N$.
%
Disjunctive case splitting is handled by the two rules
\semlab{Case$_1$} and \semlab{Case$_2$}. The former rule handles the
success case, when the scrutinee's tag $\ell$ matches the tag of the
success clause, thus binds the payload $V$ to $x$ and proceeds to
evaluate the continuation $M$. The latter rule handles the
fall-through case, here the scrutinee gets bounds to $y$ and
evaluation proceeds with the continuation $N$.
%
The \semlab{Let} rule eliminates a trivial computation term
$\Return\;V$ by substituting $V$ for $x$ in the continuation $N$.
%

\paragraph{Evaluation contexts}
Recall from Section~\ref{sec:base-language-terms},
Figure~\ref{fig:base-language-term-syntax} that the syntax of let
bindings allows a general computation term $M$ to occur on the right
hand side of the binding, i.e. $\Let\;x \revto M \;\In\;N$. Thus we
are seemingly stuck in the general case, as the \semlab{Let} rule only
applies if the right hand side is a trivial computation.
%
However, it is at this stage we make use of the notion of
\emph{evaluation contexts} due to \citet{Felleisen87}. An evaluation
context is syntactic construction which decompose the dynamic
semantics into a set of base rules (i.e. the rules presented thus far)
and an inductive rule, which enables us to focus on a particular
computation term, $M$, in some larger context, $\EC$, and reduce it in
the said context to another computation $N$ if $M$ reduces outside out
the context to that particular $N$. In our formalism, we call this
rule \semlab{Lift}. Evaluation contexts are generated from the empty
context $[~]$ and let expressions $\Let\;x \revto \EC \;\In\;N$.

\section{Row polymorphism}
\label{sec:row-polymorphism}

\section{Type and effect inference}
\dhil{While I would like to detail the type and effect inference, it
  may not be worth the effort. The reason I would like to do this goes
  back to 2016 when Richard Eisenberg asked me about how we do effect
  inference in Links.}

\chapter{Unary handlers}
\label{ch:unary-handlers}

\section{Deep handlers}
\subsection{Syntax and static semantics}
\subsection{Effect inference}
\subsection{Dynamic semantics}

\section{Default handlers}

\section{Parameterised handlers}

\section{Shallow handlers}
\label{ch:shallow-handlers}

\subsection{Syntax and static semantics}
\subsection{Dynamic semantics}

\chapter{N-ary handlers}
\label{ch:multi-handlers}

% \section{Syntax and Static Semantics}
% \section{Dynamic Semantics}
\section{Unifying deep and shallow handlers}

\part{Implementation}

\chapter{Continuation passing styles}
\chapter{Abstract machine semantics}

\part{Expressiveness}
\chapter{Computability, complexity, and expressivness}
\label{ch:expressiveness}
\section{Notions of expressiveness}
Felleisen's macro-expressiveness, Longley's type-respecting
expressiveness, Kammar's typability-preserving expressiveness.

\section{Interdefinability of deep and shallow Handlers}
\section{Encoding parameterised handlers}

\chapter{The asymptotic power of control}
\label{ch:handlers-efficiency}
Describe the methodology\dots
\section{Generic search}
\section{Calculi}
\subsection{Base calculus}
\subsection{Handler calculus}
\section{A practical model of computation}
\subsection{Syntax}
\subsection{Semantics}
\subsection{Realisability}
\section{Points, predicates, and their models}
\section{Efficient generic search with effect handlers}
\subsection{Space complexity}
\section{Best-case complexity of generic search without control}
\subsection{No shortcuts}
\subsection{No sharing}

\chapter{Robustness of the asymptotic power of control}
\section{Mutable state}
\section{Exception handling}
\section{Effect system}

\part{Conclusions}

\chapter{Conclusions}
\label{ch:conclusions}
Some profound conclusions\dots

\chapter{Future Work}
\label{ch:future-work}

%%
%% Appendices
%%
% \appendix

%% If you want the bibliography single-spaced (which is allowed), uncomment
%% the next line.
%\nocite{*}
\singlespace
%\nocite{*}
%\printbibliography[heading=bibintoc]
\bibliographystyle{plainnat}
\bibliography{\jobname}

%% ... that's all, folks!
\end{document}
