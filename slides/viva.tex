\documentclass[169,10pt,compress,dvipsnames]{beamer}
%%
%% Slides layout
%%
\beamertemplatenavigationsymbolsempty  % hides navigation buttons
\usetheme{Madrid}                      % standard Madrid theme
\setbeamertemplate{footline}{}         % renders the footer empty
%
\setbeamertemplate{bibliography item}{ % this is a hack to prevent Madrid theme + biblatex
  \hspace{-0.4cm}\lower3pt\hbox{       % from causing bibliography entries to run over
    \pgfuseimage{beamericonarticle}    % the slide margins
}}

%%
%% Packages
%%
\usepackage[utf8]{inputenc}               % enable UTF-8 compatible typing
\usepackage{hyperref}                     % interactive PDF
\usepackage[sort&compress,square,numbers]{natbib}    % Bibliography
\usepackage{bibentry}         % Print bibliography entries inline.
\makeatletter                 % Redefine bibentry to omit hyperrefs
\renewcommand\bibentry[1]{\nocite{#1}{\frenchspacing
     \@nameuse{BR@r@#1\@extra@b@citeb}}}
\makeatother
\nobibliography*              % use the bibliographic data from the standard BibTeX setup.
\usepackage{amsmath,amssymb,mathtools}    % maths typesetting
\usepackage{../pkgs/mathpartir}           % Inference rules
\usepackage{../pkgs/mathwidth}            % renders character sequences nicely in math mode
\usepackage{stmaryrd}                     % semantic brackets
\usepackage{xspace}                       % proper spacing for macros in text

\usepackage[T1]{fontenc}                  % 8-bit font encoding
                                          % native support for accented characters.
\usepackage[scaled=0.85]{beramono}        % smoother typewriter font
\newcommand*{\Scale}[2][4]{\scalebox{#1}{\ensuremath{#2}}}%

%%
%% Calculi names.
%%
\newcommand{\Links}{Links\xspace}
\newcommand{\CoreLinks}{\ensuremath{\mathsf{CoreLinks}}\xspace}
\newcommand{\BCalc}{\ensuremath{\lambda_{\mathsf{b}}}\xspace}
\newcommand{\BCalcRec}{\ensuremath{\lambda_{\mathsf{b}+\mathsf{rec}}}\xspace}

%%
%% Calculi terms and types type-setting.
%%
\newcommand{\revto}{\ensuremath{\leftarrow}}

\newcommand{\dec}[1]{\mathsf{#1}}
\newcommand{\keyw}[1]{\mathbf{#1}}
\newcommand{\Handle}{\keyw{handle}}
\newcommand{\ShallowHandle}{\ensuremath{\keyw{handle}^\dagger}}
\newcommand{\With}{\keyw{with}}
\newcommand{\Let}{\keyw{let}}
\newcommand{\Rec}{\keyw{rec}}
\newcommand{\In}{\keyw{in}}
\newcommand{\Do}{\keyw{do}}
\newcommand{\Return}{\keyw{return}}
\newcommand{\Val}{\keyw{val}}
\newcommand{\Case}{\keyw{case}}
\newcommand{\If}{\keyw{if}}
\newcommand{\Then}{\keyw{then}}
\newcommand{\Else}{\keyw{else}}
\newcommand{\Absurd}{\keyw{absurd}}
\newcommand{\Record}[1]{\ensuremath{\langle #1 \rangle}}
\newcommand{\Unit}{\Record{}}
\newcommand{\Inl}{\keyw{inl}}
\newcommand{\Inr}{\keyw{inr}}
\newcommand{\Thunk}{\lambda \Unit.}

\newcommand{\Pre}[1]{\mathsf{Pre}(#1)}
\newcommand{\Abs}{\mathsf{Abs}}
\newcommand{\Presence}{\mathsf{Presence}}
\newcommand{\Row}{\mathsf{Row}}
\newcommand{\Type}{\mathsf{Type}}
\newcommand{\Ground}{\mathsf{ground}}

\newcommand{\Comp}{\mathsf{Comp}}
\newcommand{\Effect}{\mathsf{Effect}}
\newcommand{\Handler}{\mathsf{Handler}}

\newcommand{\ZeroType}{0}
\newcommand{\UnitType}{1}
\newcommand{\One}{1}
\newcommand{\Int}{\mathsf{Int}}
\newcommand{\Bool}{\mathsf{Bool}}
\newcommand{\List}{\mathsf{List}}
\newcommand{\Nat}{\mathsf{Nat}}
\newcommand{\Choose}{\dec{Choose}}
\newcommand{\Count}{\dec{count}}
\newcommand{\GenericSearch}{\dec{genericSearch}}
\newcommand{\Predicate}{\dec{Predicate}}
\newcommand{\Point}{\dec{Point}}
\newcommand{\Branch}{\dec{Branch}}
\newcommand{\Get}{\dec{Get}}
\newcommand{\Put}{\dec{Put}}
\newcommand{\Zero}{\dec{Zero}}
\newcommand{\Fail}{\dec{Fail}}

\newcommand{\True}{\mathsf{true}}
\newcommand{\False}{\mathsf{false}}

\newcommand{\eff}{!}
\newcommand{\typ}[2]{#1 \vdash #2}
\newcommand{\typv}[2]{#1 \vdash #2}
\newcommand{\typc}[3]{#1 \vdash #2 \eff #3}

\newcommand{\FTV}{\ensuremath{\mathrm{FTV}}}

\newcommand{\reducesto}[0]{\ensuremath{\leadsto}}
\newcommand{\stepsto}[0]{\ensuremath{\longrightarrow}}
\newcommand{\EC}{\ensuremath{\mathcal{E}}}

%% Handler projections.
\newcommand{\hret}{H^{\mathrm{val}}}
\newcommand{\hval}{\hret}
\newcommand{\hops}{H^{\mathrm{ops}}}
%\newcommand{\hex}{H^{\mathrm{ex}}}
\newcommand{\hell}{H^{\ell}}

\newcommand{\alertbox}[2]{{\par\noindent\small\color{red} \framebox{\parbox{\dimexpr\linewidth-2\fboxsep-2\fboxrule}{\textbf{#1:} #2}}}}
\newcommand{\todo}[1]{\alertbox{TODO}{#1}}
\newcommand{\dhil}[1]{\alertbox{Daniel}{#1}}

%%
%% Labels
%%
\newcommand{\slab}[1]{\textrm{#1}}
\newcommand{\klab}[1]{\textrm{K-#1}}
\newcommand{\semlab}[1]{\textrm{S-#1}}
\newcommand{\tylab}[1]{\textrm{T-#1}}
\newcommand{\mlab}[1]{\text{\scshape{M-#1}}}
\newcommand{\siglab}[1]{\text{\scshape{Sig-#1}}}
\newcommand{\rowlab}[1]{\text{\scshape{R-#1}}}

%%
%% Syntactic categories.
%%
\newcommand{\CatName}[1]{\textrm{#1}}
\newcommand{\CompCat}{\CatName{Comp}}
\newcommand{\ValCat}{\CatName{Val}}
\newcommand{\VarCat}{\CatName{Var}}
\newcommand{\TypCat}{\CatName{Type}}
\newcommand{\TyVarCat}{\CatName{TyVar}}
\newcommand{\KindCat}{\CatName{Kind}}
\newcommand{\RowCat}{\RowCat}

%%
%% Lindley's array stuff.
%%
\newcommand{\ba}{\begin{array}}
\newcommand{\ea}{\end{array}}

\newcommand{\bl}{\ba[t]{@{}l@{}}}
\newcommand{\el}{\ea}


%%
%% Lindley's syntax, reductions, equations, and derivation environments.
%%
\newenvironment{syntax}{\[\ba{@{}l@{\quad}r@{~}c@{~}l@{}}}{\ea\]\ignorespacesafterend}
\newenvironment{reductions}{\[\ba{@{}l@{\qquad}@{}r@{~~}c@{~~}l@{}}}{\ea\]\ignorespacesafterend}

\newenvironment{eqs}{\ba{@{}r@{~}c@{~}l@{}}}{\ea}
\newenvironment{equations}{\[\ba{@{}r@{~}c@{~}l@{}}}{\ea\]\ignorespacesafterend}
\newenvironment{derivation}{\begin{displaymath}\ba{@{}r@{~}l@{}}}{\ea\end{displaymath}\ignorespacesafterend}
\newcommand{\reason}[1]{\quad (\text{#1})}


\newenvironment{smathpar}{\vspace{-3ex}\small\begin{mathpar}}{\end{mathpar}\normalsize\ignorespacesafterend}

%%
%% Defined-as equality
%%
\newcommand{\defas}[0]{\mathrel{\overset{\makebox[0pt]{\mbox{\normalfont\tiny\text{def}}}}{=}}}

%%
%% Meta information
%%
\author{Daniel Hillerström}
\title{Foundations for Programming and Implementing Effect Handlers}
\institute{The University of Edinburgh, Scotland UK}
\subtitle{PhD viva}
\date{August 13, 2021}

%%
%% Slides
%%
\begin{document}

%
% Title slide
%
\begin{frame}
  \maketitle
\end{frame}

% Dissertation overview
\begin{frame}
  \frametitle{My dissertation at glance}

  Three main strands of work

  \begin{description}
  \item[Programming] Language design and applications of effect handlers.
  \item[Implementation] Canonical implementation strategies for effect handlers.
  \item[Expressiveness] Exploration of the computational expressiveness of effect handlers.
  \end{description}
\end{frame}

\begin{frame}
  \frametitle{Calculi for deep, shallow, and parameterised handlers}

  The calculi capture key aspects of the implementation of effect
  handlers in Links.

  \begin{itemize}
    \item $\HCalc$ ordinary deep handlers (fold).
    \item $\SCalc$ shallow handlers (case-split).
    \item $\HPCalc$ parameterised deep handlers (fold+state).
  \end{itemize}

  The actual implementation is the union of the three calculi.\\[2em]

  \textbf{Relevant papers} TyDe'16~\cite{HillerstromL16},
  APLAS'18~\cite{HillerstromL18}, JFP'20~\cite{HillerstromLA20}.
\end{frame}

% UNIX
\begin{frame}
  \frametitle{Effect handlers as composable operating systems}

  An interpretation of \citeauthor{RitchieT74}'s
  UNIX~\cite{RitchieT74} in terms of effect handlers.\\[2em]

  \[
    \bl
    \!\!\!\!\!\!\!\!\!\!\!\!\!\!\!\!\!\!\!\!\!\!\!\!\!\!\!\!\textbf{Basic idea}
      \ba[m]{@{\qquad}r@{~}c@{~}l}
        \text{\emph{system call}} &\approx& \text{\emph{operation invocation}}\\
        \text{\emph{system call implementation}} &\approx& \text{\emph{operation interpretation}}
      \ea
    \el
  \]\hfill\\[2em]

  \textbf{Key point} Legacy code is modularly retrofitted with functionality.
\end{frame}

% CPS translation
\begin{frame}
  \frametitle{CPS transforms for effect handlers}

  A higher-order CPS transform for effect handlers with generalised
  continuations.\\[1em]

  \textbf{Generalised continuation} Structured representation of
  delimited continuations.\\[0.5em]

  \[
    \Scale[1.8]{\kappa = \overline{(\sigma, (\hret,\hops))}}
  \]\\[1em]

  \textbf{Key point} Separate the \emph{doing} layer ($\sigma$) from the \emph{being} layer ($H$).\\[2em]

  \textbf{Relevant papers} FSCD'17~\cite{HillerstromLAS17},
  APLAS'18~\cite{HillerstromL18}, JFP'20~\cite{HillerstromLA20}.
\end{frame}

% Abstract machine
\begin{frame}
  \frametitle{Abstract machine semantics for effect handlers}

  Plugging generalised continuations into \citeauthor{FelleisenF86}'s
  CEK machine~\cite{FelleisenF86} yields a runtime for effect
  handlers.\\[2em]

  \[
    \Scale[2]{\cek{C \mid E \mid K = \overline{((H,E), \sigma)}}}
  \]\\[2em]

  \textbf{Relevant papers} TyDe'16~\cite{HillerstromL16},
  JFP'20~\cite{HillerstromLA20}.

\end{frame}

% Interdefinability of handlers
\begin{frame}
  \frametitle{Interdefinability of effect handlers}

  Deep, shallow, and parameterised handlers are interdefinable
  w.r.t. to typability-preserving macro-expressiveness.

  \begin{itemize}
    \item Deep as shallow, $\mathcal{D}\llbracket - \rrbracket$, image is computationally lightweight.
    \item Shallow as deep, $\mathcal{S}\llbracket - \rrbracket$, image is computationally expensive.
    \item Parameterised as deep, $\mathcal{P}\llbracket - \rrbracket$,
      image uses explicit state-passing.
    \end{itemize}
    ~\\[1em]
    \textbf{Relevant papers} APLAS'18~\cite{HillerstromL18},
    JFP'20~\cite{HillerstromLA20}.

\end{frame}

% Asymptotic speed up with first-class control
\begin{frame}
  \frametitle{Asymptotic speed up with effect handlers}

  Effect handlers can make some programs faster!

  \[
    \Count_n : ((\Nat_n \to \Bool) \to \Bool) \to \Nat
  \]\\[1em]
  %
  Using type-respecting expressiveness
  \begin{itemize}
  \item There \textbf{exists} an implementation of $\Count_n \in \HPCF$ with
    effect handlers such that the runtime for every $n$-standard predicate $P$ is
    $\Count_n~P = \BigO(2^n)$.
  \item \textbf{Forall} implementations of $\Count_n \in \BPCF$ the runtime for every $n$-standard predicate $P$ is $\Count_n~P = \Omega(n2^n)$
  \end{itemize}
  ~\\[1em]
  \textbf{Relevant paper} ICFP'20~\cite{HillerstromLL20}.
\end{frame}

% Background
% \begin{frame}
%   \frametitle{Continuations literature review}
% \end{frame}

%
% References
%
\begin{frame}%[allowframebreaks]
  \frametitle{References}
  \bibliographystyle{plainnat}
  \bibliography{\jobname}
\end{frame}
\end{document}